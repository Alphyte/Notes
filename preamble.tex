\usepackage{tikz}
% \usepackage{xcharter-otf}
\usepackage{amsthm, thmtools}
\usepackage[margin=0.5in]{geometry}
\usepackage{mathtools}
\usepackage{physics}
\usepackage{tikz}
\usepackage{pgfplots}
\usepackage{unicode-math}
\usepackage{xcharter-otf}
\usepackage[cal=esstix,calscaled=1.02]{mathalpha}
% \DeclareFontFamily{U}{stixtwoplain}{\skewchar\font =127}
%   \DeclareFontShape{U}{stixtwoplain}{m}{n} {
% 	<-> \mathalfa@calscaled stix2-mathcal}{}
% \usepackage{hyperref}
% \usepackage[dvipsnames]{xcolor}
% \SetMathAlphabet{\mathcal}{bold}{U}{stixtwoplain}{b}{n}


\geometry{margin=3in}
\usepackage[%
  colorlinks = true,
  citecolor  = RoyalBlue,
  linkcolor  = RoyalBlue,
  urlcolor   = RoyalBlue,
  unicode]{hyperref}

% \setmainfont{EB Garamond}
% \setmathfont{Garamond Math}

\declaretheoremstyle[
    spaceabove=5pt, spacebelow=5pt,
    headfont=\normalfont\bfseries,
    notefont=\mdseries, notebraces={(}{)},
    bodyfont=\normalfont,
    postheadspace=1em
]{tstyle}

\declaretheorem[style=tstyle, numberwithin=chapter]{theorem}
\declaretheorem[style=tstyle, sibling=theorem]{lemma, proposition, corollary}

\declaretheorem[style=tstyle, numberwithin=chapter]{definition}
\declaretheorem[style=tstyle, numberwithin=chapter]{example}


\newcommand{\bb}{\mathbb}
\newcommand{\mc}{\mathcal}
\newcommand{\ms}{\mathscr}
\newcommand{\ol}{\overline}
\newcommand{\wh}{\widehat}
\newcommand{\mrm}{\mathrm}

\newcommand{\veps}{\varepsilon}

\DeclareMathOperator{\supp}{supp}
% \DeclareMathOperator{\im}{im}
\DeclareMathOperator{\carrier}{carrier}