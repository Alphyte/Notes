\chapter{Energy}

\section{Kinetic Energy and the Work-Energy Theorem}

The \nt{kinetic energy} $T$ of a particle is given by the quadratic form $m\dot r^2/2$. The derivative of the kinetic energy is given by $\dot T = m\dot r \cdot \ddot r = F(r) \cdot \dot r$. Integrating this on both sides, it can be seen that the change of kinetic energy as a particle moves along a path $r(t)$ from time $t_0$ to $t_f$ while subjected to a force $F(r)$ is equal to the \nt{work} done on that particle, \[ \int_{r(t_0)}^{r(t_f)} F(r) \dd{r}.\] This equivalence is called the \nt{work-energy theorem}.

\section{Potential Energy}

Suppose we have a system governed by a force field $F(r)$. We call $F$ a \nt{conservative force field} if $F$ has a potential, that is a function $U(r)$ such that $F = -\grad U$. Poincare's lemma tells us that this is equivalent to the condition that $\curl F = 0$. 

The function $U$ is called \nt{potential energy} of the system. It is easy to see that in one dimension, there is always a potential, namely $U(r) = -\int_c^r F(r) \dd{r}$, where $c$ is any constant. In higher dimensions, it is necessary to verify that the curl is zero. 

For conservative fields it can be seen that the equations of motion can be rewritten as $\ddot r = - \grad U(r)$. 

\begin{example}[Gravity near the surface of the earth]
    Gravity on a particle on mass $m$ near the surface of the earth can be described as a force field $F(h) = -mg$, which results in a potential dependant on the height $h$ of the particle $U(h) = mgh$.
\end{example}

Why is the negative sign there? If we consider the physical system governed only by gravity, $F(r) = -g$, we want a particle at a higher point in space to have more ``potential,'' as it can accumulate more kinetic energy as it falls.

In fact, for any system governed by a conservative force field, it holds that $T + U$ remains constant. This is because \[\dv{t} T + U = F(r) \cdot \dot r + -F(r) \cdot \dot r = 0 \] by application of the work-energy theorem and the chain rule. This result is called the \nt{conservation of energy}.

