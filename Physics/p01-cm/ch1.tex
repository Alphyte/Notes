\chapter{Newton's Formalism of Mechanics}

Mechanics is the study of motion of objects. This started with the Greeks, but their ideas are antiquated and of little use. Newton and Galileo were the first to do mechanics as we know it today, and Newton's formulation is taught in introductory physics courses. However, two alternative formulations, Lagragian and Hamiltonian mechanics, are equivalent to Newtonian mechanics, and sometimes provides more elegant solutions to problems.

In this chapter, we will discuss only Newton's formalism of mechanics.

Classical mechanics is about studying the motion of macroscopic particles when $v \ll c$. 

\section{Space and Time}

In this section, we follow the approach of Arnold to describe 

Space can be described $\bb A^3$, affine 3-dimensional space. After picking a  origin and orthonormal basis, we can induce a bijection $\bb A^3 \to \bb R^3$. We can carry over the standard inner product structure on $\bb R^3$ as well.

Time is can be described as $\bb A^1$, affine one-dimensional space. After picking an origin and a basis, similarly to space, we can induce a bijection $\bb A^1 \to \bb R^1$.

Now, a \nt{reference frame} is just a choice of these origins and bases, and most of the time when we do physics, we either implicity or explicitly define a reference frame. 

\section{Newton's First Law}

Newton's first law isn't so much a law as it is a definition.