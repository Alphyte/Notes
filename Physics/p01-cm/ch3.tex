\chapter{Motion of a Single Particle}

\section{Phase Portraits}



\section{Projectiles}

In the absence of forces other than gravity, the 2D equations of motion of a particle according to Newton's laws are $\ddot x = 0, \ddot y = -g$, where $g \approx \SI{9.8}{\newton\per\kilogram} = \SI{9.8}{\meter\per\second\squared}$ is an approximation of the gravitational field near the surface of the Earth.

From this we can derive the equations $x(0) = x(0) + \dot x(0)t$, $y = y_0 + \dot y(0) - gt^2/2$. From there we can derive the range formula and other properties about this type of motion.

\section{Resistive Forces}

Resistanve isn't always simple to describe: The exact force caused by a resistive medium is often a complicated function of the velocity $f(v)$, but here we are only concerned with the lower-order terms.

\subsection{Linear Mediums}

Here, we consider physical systems where the resistive force $f(v) \approx -kv$. These 

\subsection{Quadratic Mediums}

\section{Charged Particle in a Magnetic Field}

The force on charged particle in a magnetic field is described by the Lorentz law $F = qv \times B$. 

\section{Oscillations}

