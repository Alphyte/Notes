\chapter{Calculus of Variations}

\section{Basic Examples}

\subsection*{Geodesic}

In Euclidean space, the length of a curve $x(t), y(t)$ is defined as \[\int_0^{t_f} \sqrt{x'(t) + y'(t)} \dd{t}.\] To find the geodesic between two points, i.e. the shortest curve between two points $P$ and $Q$, we consider the constraints $(x(0), y(0)) = p$ and $(x(t_f), y(t_f)) = q$.

In general, on a Riemannian manifold $(M, g)$, you can find the geodesic connecting two points $p$ and $q$ by minimizing $\int_0^{t_f} \norm{\gamma'(t)}_g^2 \dd{t}$ subject the constraints $\gamma(0) = p$, $\gamma(1) = q$.

\subsection*{Catenary}

Consider a bridge with a cable hanging from two posts. We want to find the shape of the cable, which is equivalent to minimizing the potential energy of the cable.

If the height of cable at position $x$ is denoted by $h(x)$ and the cable's linear density is denoted by $\lambda$, then this is equivalent to minimizing \[\int_0^\ell \lambda g h(x)\sqrt{1 + h'(x)} \dd{x}\] with the constraints $h(0) = h(\ell) = h_0$.

One can see that both examples involve constrained minimation of a \textit{functional}.

\section{Basic Techniques}

In both cases, we have a functional $\Phi \colon F \to \bb R$, of the form $\int L(q, \dot q, t)$.