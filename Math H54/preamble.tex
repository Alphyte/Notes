\usepackage{tikz}
% \usepackage{xcharter-otf}
\usepackage{amsthm, thmtools}
\usepackage{tikz-cd}
\usepackage{physics}
\usepackage{tikz}
\usepackage{pgfplots}
\usepackage{esvect}
\usepackage[%
  colorlinks = true,
  citecolor  = RoyalBlue,
  linkcolor  = RoyalBlue,
  urlcolor   = RoyalBlue,
  unicode,
  ]{hyperref}
\usepackage{unicode-math}


\usetikzlibrary {arrows.meta, positioning}

\setmainfont{EB Garamond}
\setmathfont{Garamond Math}

\declaretheoremstyle[
    spaceabove=10pt, spacebelow=10pt,
    headfont=\normalfont\bfseries,
    notefont=\mdseries, notebraces={(}{)},
    bodyfont=\normalfont,
    postheadspace=1em
]{tstyle}

\declaretheorem[style=tstyle, numberwithin=section]{theorem}
\declaretheorem[style=tstyle, sibling=theorem]{lemma, proposition, corollary}

\declaretheorem[style=tstyle, numberwithin=section]{definition}
\declaretheorem[style=tstyle, numberwithin=section]{example}


\newcommand{\bb}{\mathbb}
\newcommand{\mc}{\mathcal}

\newcommand{\introtext}[1]{\begin{center}
  \begin{minipage}{0.5\textwidth}
    \begin{small}
      #1
    \end{small}
  \end{minipage}
  \vspace{0.5cm}
\end{center}}