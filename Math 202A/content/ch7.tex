\chapter{Integration of Functions}

Now that we are able to measure the sizes of sets, we aim to define the integral of functions $f \colon (X, \mc S, \mu) \to \mc B$, where $\mc B$ is a Banach space. The most important example will be when $\mc B = \bb R$. In this section, $X$ will denote a measure space with $\sigma$-ring $\mc S$ and measure $\mu$, and $\mc B$ will denote a Banach space over $\bb K$.

\section{Integrating Simple Functions}

From intuition, we want the integral of the indicator function of a set $A$ to be the size (i.e. measure) of that set. However, when the sets are not measurable, this won't work, so we restrict ourselves to indicator functions of measurable sets.

Also, we want the integral to be linear. Through these two properties, we can define the integral of linear combinations of indicator functions of measurable sets.

\begin{definition}
    We say $f \colon X \to \mc B$ is a \textbf{simple measurable function} if the image of $f$ is finite and if $f^{-1}\qty(\qty{y})$ is measurable for any $y \in \mc B$.
\end{definition}

\begin{lemma}
    Any simple measurable function can be written as \[f = \sum_{y \in \im f} y\chi_{f^{-1}(\{y\})}.\]
\end{lemma}

\begin{lemma}
    A function $f\colon X \to \mc B$ is simple measurable iff it can be written as \[f = \sum_{i= 1}^n y_i\chi_{E_i}\] where $y_i \ne 0$ and are distinct, and the $E_i$ are disjoint and nonempty.

    Furthermore, if $f$ can be written in such a form, then the $y_i$ must range over $\im(f) \setminus \qty{0}$, and the $E_i$ must be the sets $f^{-1}(\{y_i\})$.
\end{lemma}

\begin{proposition}
    The simple measurable functions from $X \to \mc B$ form a $\bb K$-vector space. In fact, if $\mc B$ is a Banach algebra, they form a $\bb K$-algebra.
\end{proposition}

\begin{proof}
    Clearly $kf$ is simple measurable since $\im(kf) = k\im(f)$, and $(kf)^{-1}(ky) = f^{-1}(y)$.

    Thus, if $f$ and $g$ are simple measurable, then let \[f = \sum_{i = 1}^m a_i\chi_{A_i}\quad \text{and}\quad g = \sum_{i = 1}^n b_i\chi_{B_i}.\] as in the previous lemma.

    For convinience, let $A_{m + 1} = X \setminus \bigcup A_i$, $B_{n + 1} = X \setminus \bigcup A_j$, and $a_{m + 1} = 0 = b_{m + 1}$. Then, let $C_{ij} = A_i \cap B_j$ and $c_{ij} = a_i + b_j$. We can write \[f + g = \sum_{i = 1}^{m + 1} \sum_{j = 1}^{n + 1} c_{ij}\chi_{C_{ij}}.\] 
    
    However, the $c_{ij}$ might be zero and the $C_{ij}$ might be empty, so we remove those terms from the summation. In addition, if there are two $c_{ij}$s that are equal, we combine those terms in the summation. After doing these steps, we have a summation in the form required by the previous lemma, and thus we are done.

    The proof for multiplication when $\mc B$ is an Banach algebra is similar.
\end{proof}

\begin{corollary}
    The simple measurable functions are the span of the indicator functions of measurable sets.
\end{corollary}

Finally, we define the integral for simple measurable functions using the rules we stated at the beginning of the chapter.

\begin{definition}
    For a simple measurable function $f \colon X \to \mc B$, we define the \textbf{integral} of $f$ to be \[\int f \dd{\mu} = \sum_{y \in \im(f) \setminus \qty{0}} \mu(f^{-1}(\{y\}))y.\]
\end{definition}

Clearly $\int \chi_E \dd{\mu} = \mu(E)$, and satisfies the two properties we proposed at the beginning of this section:

\begin{proposition}
    $\int \chi_E \dd{\mu} = \mu(E)$, and the integral is $\bb K$-linear.
\end{proposition}

\section{Measurable Functions}

Integrating simple functions isn't terribly interesting though, so we want to extend integration to functions that are non-simple. We can do this by considering the class of functions that are well approximable by SMFs, which turns out to be incredibly rich.

\begin{definition}
    If $f \colon X \to \mc B$, (where $\mc B$ could be just a pointed metric space), we say $f$ is \textbf{$\mc S$-measurable} if there exists a sequence $\qty{f_n}$ of SMFs that converges pointwise to $f$.

    If we have a measure $\mu$ on $\mc S$, the null sets form a $\sigma$-ring. In this situation, we say that $f$ is \textbf{$\mu$-measurable} if there exists a sequence $\qty{f_n}$ that converges to $f$ $\mu$-\emph{almost everywhere}, i.e., there exists a null set $N$ such that $f_n \to f$ on $X \setminus N$.
\end{definition}

To make the theory work smoothly, however, we need the following non-obvious fact:

\begin{proposition}
    If $f_n \to f$, where the functions $f_n$ are $\mc S$-measurable, then $f$ is measurable.
\end{proposition}

We will prove this later, but first we try and isolate properties of measurable functions to make this easier to prove.

\begin{proposition}
    If $\qty{f_n}$ are SMFs, and $f_n \to f$ pointwise, then \[\im(f) = \ol{\qty(\bigcup_{i = 1}^\infty \im(f_i))},\] and thus $\im(f)$ is separable, i.e. is contained in a closure of a countable set.
\end{proposition}

\begin{proposition}
    The carrier of a measurable function $f$ is contained in the union of the carriers of $f_n$, which are SMFs converging to $f$, and thus there is a set $E \in \mc S$ where $\carrier(f) \subseteq E$.
\end{proposition}

Now we get the main property
\begin{proposition}
    If $f$ is measurable then if $U$ is open in $\mc B$, we have $f^{-1}(U) \cap E \in \mc S$.
\end{proposition}

\begin{proof}
    $x \in f^{-1}(U)$ iff $f(x) \in U$. Now, we define $U_m$ to be the set $\qty{x \in U \mid d(x, U') > 1/m}$. Then, we have $\ol{U_m} \subseteq U_{m - 1}$. We can see that $f(x) \in U$ iff there exists $n$ such that $f(x) \in U_n$. Now, if $f_k \to f$ is any sequence of functions converging to $f$ pointwise, then using approximation and the triangle inequality, we see that $f(x) \in U$ if and only if there exists a $n$ and $K$ such for $k \ge K$ such that $f_k(x) \in U_n$.

    Now we have \[x \in f^{-1}(U) \cap E\iff x \in \bigcup_{n = 1}^\infty \bigcup_{K = 1}^\infty \bigcap_{k = K}^\infty f_k^{-1}(U_n) \cap E,\] thus $f^{-1}(U) \cap E \in \mc S$ for $f_k$ simple.
\end{proof}

\begin{corollary}
    If $f$ is measurable, for any Borel set $A \subseteq \mc B$, then $f^{-1}(A) \in \mc S$. \textbf{This is the traditional definition of measurable}.
\end{corollary}

\begin{corollary}
    If $f_n$ is a sequence of measurable functions converging to $f$ then $f$ satisfies
    \begin{enumerate}
        \item $\im(f)$ is separable.
        \item There is a set $E \in \mc S$ where $\carrier(f) \subseteq E$.
        \item If $f_n$ is a sequence of measurable functions and $f_n \to f$, then $f^{-1}(A) \cap E\in \mc S$ for all Borel sets $A \subseteq \mc B$. 
    \end{enumerate}
\end{corollary}

Now, to prove that limits of measurable functions are measurable, we just need to prove that functions that satisfy 1, 2, and 3 are measurable.

\begin{theorem}
    If $f \colon X \to \mc B$ has properties 1, 2, and 3, then it is measurable. That is, a function $f$ is a $\mc S$-measurable function if and only if \begin{enumerate}
        \item $\im(f)$ is separable.
        \item There is a set $E \in \mc S$ where $\carrier(f) \subseteq E$.
        \item If $f_n$ is a sequence of measurable functions and $f_n \to f$, then $f^{-1}(A) \cap E\in \mc S$ for all Borel sets $A \subseteq \mc B$. 
    \end{enumerate}
\end{theorem}

\begin{proof}
    In this proof, we will work implicity with $X = E$ to avoid many intersections. Now, to construct a sequence of SMFs that converge to $f$, we enumerate a sequence $\qty{b_j}$ that is dense in $\im(f)$. Set $C_{ji} = B_{1/j}(b_i)$, whose preimage is in $\mc S$, and we order the pairs $(j, i)$ lexicographically. Now, for fixed $n$, we construct $f_n$ as follows: disjointize the $C_{ji}$ for $j, i \le n$ in reverse of the order given. Thus, $E_{nn} = C_{nn}$, and $E_{ij} = C_{ji} \setminus \bigcup \qty{ C_{\ell k} \mid ji < \ell k \le nn, i \le n}$. Now we let $f_n = \sum_{ji \le nn, i \le n} b_i \chi_{E_{ji}}$.

    Now we prove that $f_n \to f$ pointwise. If we fix $x \in X$ and $\veps > 0$, there exists some point $b_N$ where $\norm{f(x) - b_N} < \veps$, and thus $x$ is contained within $C_{NK}$ for $K = \lceil 1/\veps \rceil$. Now, if we take $M = \max(N, K)$, then we can see for all $m \ge M$, $f_m(x)$ is either $b_N$, or some other $b$ such that $\norm{f(x) - b} < 1/M < \veps$.
\end{proof}

Here are some standard facts about measurable functions:

\begin{proposition}
    \begin{enumerate}
        \item The set of $\mc S$-measurable functions form a vector space.
        \item If $f$ is measurable, then $\norm{f}$ is a measurable function from $X \to \bb R$.
        \item If $f$ is measurable and $g$ is a real-valued measurable function, then $gf$ is measurable. 
    \end{enumerate}
\end{proposition}

\section{Modes of Convergence}

Now that we have developed the theory of measurable functions, we go back to integration.

We first define what it means for a simple function to be integrable:

\begin{definition}
    A SMF $f$ is \textbf{simple integrable} iff $\int \norm{f} \dd{\mu} < \infty$, which is equivalent to $\mu(E) < \infty$ for all $E = f^{-1}(\qty{y})$ for $y \in \im(f) \setminus \qty{0}$.
\end{definition}

Considering only SIFs, we see the following properties:

\begin{proposition}
    \begin{enumerate}
        \item $f \mapsto \int f\dd{\mu}$ is linear.
        \item If $f$ is a real-valued SIF where $f \ge 0$, then $\int f\dd{\mu} \ge 0$.
        \item The function \[f \mapsto \int \norm{f} \dd{\mu}\] is a seminorm.
        \item The functions $f$ such that $\int \norm{f} \dd{\mu} = 0$ are equal to $0$ a.e.
    \end{enumerate}
\end{proposition}

We want to take the completion of this space and extend the integral operator.

However, we have bad news: A sequence $\qty{f_n}$ of SIF that is Cauchy for $\norm{\cdot}_1$ need not converge pointwise at any point.

\begin{example}
    Let $X = [0, 1]$ and $\mu$ be the Lebesgue measure, then we let $f_1 = \chi_{[0, 1/2]}$, $f_2 = \chi_{[1/2, 1]}$, and continue the sequence as so: $\chi_{[0, 1/3]}, \chi_{[1/3, 2/3]}$, $\chi_{[2/3, 1]}$, $\chi_{[0, 1/4]}$, and so on.

    This sequence is Cauchy in mean (Cauchy in the $L^1$ norm), but does not converge pointwise, as it is not Cauchy anywhere.
\end{example}

\begin{definition}
    If $f_n$ is a sequence of measurable functions and $f$ is measurable, then $f_n$ \textbf{converges to $f$ in measure} if for all $\veps > 0$, we have $\mu\qty(\qty{x \mid \norm{f_n(x) - f(x)} \ge \veps}) \to 0$.
\end{definition}

\begin{definition}
    If $f_n$ is a sequence of measurable functions then $f_n$ is \textbf{Cauchy in measure} if for all $\veps > 0$, we have $\sup_{n, m \ge N}\mu\qty(\qty{x \mid \norm{f_n(x) - f_m(x)} \ge \veps})$ converges to $0$ as $N \to \infty$.
\end{definition}

\begin{proposition}
    If a sequence if SIFs $f_n$ is Cauchy in the $L^1$ (or $L^p$) norm, then it is Cauchy in measure.
\end{proposition}

\begin{proof}
    For a given $\veps > 0$, and $m,n \in \bb N$, then we let $E_{mn}^\veps = \qty{x \in X \mid \norm{f_m(x) - f_n(x)} \ge \veps}$. We note that $\veps \chi_{E_{mn}^\veps} \le \norm{f_m(x) - f_n(x)}$. Now we have $\int \norm{f_m(x) - f_n(x)} \dd{\mu} = \norm{f_n - f_m} \ge \veps\mu(E_{mn}^\veps)$. Thus, we have that $\mu(E_{mn}^\veps) \to 0$ as $m, n \ge N$ and $N \to \infty$.
\end{proof}

\begin{corollary}[Chebyshev's inequality]
    If $X$ is a random variable, then if $\mu := \bb E[X], \sigma^2 := \bb E[(X - \mu)^2]$, then $\bb P[\abs{X - \mu} \ge \veps] \le \sigma^2/\veps$.
\end{corollary}

We prove that convergence in measure is unique in a sense.

\begin{proposition}
    If $f_n \to f, g$ where all functions are measurable, then $f = g$ almost everywhere.
\end{proposition}

\begin{proof}
    $\norm{f(x) - g(x)} \le \norm{f(x) - f_n(x)} + \norm{f_n(x) - g(x)}$ by the triangle inequality. Then, we also have \[\mu(\qty{x \mid \norm{f(x) - g(x)} \ge \veps}) \le \mu\qty(\qty{x \mid \norm{f(x) - f_n(x)} \ge \veps/2}) + \mu\qty(\qty{x \mid \norm{f(x) - f_n(x)} \ge \veps/2})\] where the RHS converges to $0$ due to definition of measure. Thus we have that the LHS converges to $0$ for all $\veps > 0$. If we let $E^\veps = \qty{x \mid \norm{f(x) - g(x)}} \ge \veps$, then we see that the set $E = \qty{x \mid f(x) \ne g(x)} = \bigcup_n E^{1/n}$. Due to countable additivity, $\mu(E) \le \sum E^{1/n} = 0$.
\end{proof}

We also have some pretty standard properties:

\begin{proposition}
    If $f_n \to f$ and $g_n \to g$ in measure, then $af_n + bg_n \to af + bg$ in measure. Furthermore, $x \mapsto \norm{f_n(x)}$ converges to $x \mapsto \norm{f(x)}$ in measure.
\end{proposition}

\begin{definition}
    If $f_n$ is any sequence of $\mc B$-valued functions, we say that $f_n \to f$ \textbf{almost uniformly} if for every $\veps > 0$ there exists a set $E^\veps \in \mc S$ such that $\mu(E^\veps) < \veps$, and $f_n \to f$ uniformly on $X \setminus E^\veps$.

    $f_n$ is \textbf{almost uniformly Cauchy} if for all $\veps > 0$ there exists a set $E^\veps \in \mc S$ such that $\mu(E^\veps) < \veps$, and $f_n$ is uniformly Cauchy on $X \setminus E^\veps$.
\end{definition}


\begin{theorem}[Riesz-Weyl]
    If $f_n$ is a sequence of $\mc S$-measurable functions that is Cauchy in measure, then there is a subsequence of $f_n$ that is almost uniformly Cauchy.
\end{theorem}

\begin{proof}
    Consider the sequence $n_k$ defined by $\mu(\qty{x \mid \abs{f_{n_k} - f} \ge 2^{-k}}) < 2^{-k}$.
\end{proof}

\section{The General Integral}

\begin{proposition}
    If $f_n$ is a sequence of measurable functions that converges to $f$ almost uniformly, then it converges to $f$ in measure.
\end{proposition}

\begin{theorem}
    Let $f$ be a measurable function, then the following are equivalent:
    \begin{enumerate}
        \item There is a sequence of SIFs $f_n \to f$ in measure, such that $f_n$ is Cauchy in mean.
        \item There is a sequence of SIFs $g_n \to f$ almost uniformly, such that $g_n$ is Cauchy in mean.
        \item There is a sequence of SIFs $h_n \to f$ almost everywhere, such that $h_n$ is Cauchy in mean.
    \end{enumerate}
\end{theorem}

\begin{proof}
    We prove this as a chain of implications:
    \begin{enumerate}
        \item[$(1) \Rightarrow (2)$] By the Riesz-Weyl theorem, we have that some subsequence $g_n$ of $f_n$ that is almost uniformly Cauchy, which implies that  it converges almost uniformly to some function $g$. Since by the previous proposition that implies it converges in measure to $g$, that must mean $g = f$ a.e. by Proposition 7.17, which implies that $g_n \to f$ almost uniformly.
        \item[$(2) \Rightarrow (3)$] Consider $A^{1/n}$ such that $f_n \to f$ uniformly on $A^{1/n}$ with $\mu(X \setminus A^{1/n}) < 1/n$. Then, $f_n \to f$ on $\bigcup A^{1/n}$.
        \item[$(3) \Rightarrow (1)$] Apply the Riesz-Weyl theorem to get a sequence $g_n$ of $h_n$ that is almost uniformly Cauchy, which implies that converges almost uniformly to $f$. Now, it is clear that for any $\delta, \veps$, we can choose $A$ where $\mu(X \setminus A) < \veps$, and $g_n \to f$ uniformly on $A$. In addition, we can choose $N$ large enough that $\norm{g_n(x) - f(x)} < \delta$ for all $x \in A$, which means that \[\mu\qty(\qty{x \mid \norm{g_n(x) - f(x)} \ge \delta}) < \veps\] for all $n \ge N$.
    \end{enumerate}
\end{proof}

\begin{definition}
    We say $f_n$ is \textbf{integrable} if one of these conditions holds. The set of integrable functions from $(X, \mc S, \mu) \to \mc B$ is denoted by $\mc L^1(X, \mc S, \mu, \mc B)$.
\end{definition}

We can note that $\mc L^1(X, \mc S, \mu, \mc B)$ is a module over $\mc L^\infty(X, \mc S, \mu, \bb K)$, which is the set of essentially bounded functions.

% \begin{proposition}
%     If $f_n$ is Cauchy in mean and some subsequence of it converges to $f$ in mean, then $f_n \to f$ in mean.
% \end{proposition}

To properly complete the space of SIFs under the $L^1$ norm, we want some sort of uniqueness property of convergence.

\begin{proposition}
    If the sequences $f_n$ and  $g_n$ are Cauchy in mean and converge to $f$ and $g$ respectively, where $\norm{f_n -g_n}_1 \to 0$, then $f = g$ a.e.
\end{proposition}

When we define the integral of a integrable function $f$, we want to approximate it with SIFs. The following proposition states that it does not matter which sequence of SIFs we wish to use to approximate $f$ with.


\begin{definition}
    If $f$ is an integrable function, then we define the \textbf{integral} of $f$, \[\int f \dd{\mu} = \lim_{n \to \infty} \int f_n \dd{\mu},\] where $f_n$ is a mean-Cauchy sequence of SIFs converging in measure to $f$.
\end{definition}

The following proposition implies that this value does not depend on the choice of sequence, and thus the definition above is a valid one.

\begin{proposition}
    If $f_n$ and $g_n$ are mean Cauchy sequences of SIFs where both converge to $f$ in measure. Then, they are equivalent, i.e. $\norm{f_n \to g_n}_1 \to 0$.
\end{proposition}

\begin{proof}
    We first prove that if $h_n \to 0$ in measure and is mean Cauchy, then it converges to $0$ in mean. By the Riesz-Weyl theorem, there exists a subsequence $j_n$ that is almost uniformly Cauchy. Let $E_n = \mathrm{carrier}(j_n)$, and $E = \bigcup E_n$. Then, for $N$ such that $\norm{j_n - j_m} < \veps/4$ when $n, m \ge N$, we have that \[\int_{E \setminus E_N} \norm{j_n} \dd{\mu} = \norm{j_n - j_N}_1 < \veps/4\] and we have that $j_n$ almost uniformly converges to $0$ on $E_N$, which has finite measure.

    Now, we choose $G$ such that $j_n \to 0$ uniformly on $E_N \setminus G$, where $\mu(G) < \veps/(4 + 4\norm{g_N}_\infty)$, and thus \[\int_{E_N \setminus G} j_n \to 0.\] Now, on $G$, we have \[\int_G \norm{j_n} \dd{\mu} \le \int_G \norm{j_n - j_N} + \norm{j_N} \dd{\mu}\le \veps/4 + \mu(G)\norm{f_N}_\infty < \veps/2,\] and thus picking $N' \ge N$ big enough where $\norm{j_n} < \veps/(4\mu(E_N \setminus G))$ for all $n \ge N'$, and we have that $\norm{j_n}_1 < \veps$ for all $n \ge N'$.    

    Then, it suffices to see that $f_n - g_n \to 0$ in measure.
\end{proof}

