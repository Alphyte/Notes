\chapter{Integration of Functions}

Now that we are able to measure the sizes of sets, we aim to define the integral of functions $f \colon (X, \mc S, \mu) \to \mc B$, where $\mc B$ is a Banach space. The most important example will be when $\mc B = \bb R$. In this section, $X$ will denote a measure space with $\sigma$-ring $\mc S$ and measure $\mu$, and $\mc B$ will denote a Banach space over $\bb K$.

\section{Integrating Simple Functions}

From intuition, we want the integral of the indicator function of a set $A$ to be the size (i.e. measure) of that set. However, when the sets are not measurable, this won't work, so we restrict ourselves to indicator functions of measurable sets.

Also, we want the integral to be linear. Through these two properties, we can define the integral of linear combinations of indicator functions of measurable sets.

\begin{definition}
    We say $f \colon X \to \mc B$ is a \textbf{simple measurable function} if the image of $f$ is finite and if $f^{-1}\qty(\qty{y})$ is measurable for any $y \in \mc B$.
\end{definition}

\begin{lemma}
    Any simple measurable function can be written as \[f = \sum_{y \in \im f} y\chi_{f^{-1}(\{y\})}.\]
\end{lemma}

\begin{lemma}
    A function $f\colon X \to \mc B$ is simple measurable iff it can be written as \[f = \sum_{i= 1}^n y_i\chi_{E_i}\] where $y_i \ne 0$ and are distinct, and the $E_i$ are disjoint and nonempty.

    Furthermore, if $f$ can be written in such a form, then the $y_i$ must range over $\im(f) \setminus \qty{0}$, and the $E_i$ must be the sets $f^{-1}(\{y_i\})$.
\end{lemma}

\begin{proposition}
    The simple measurable functions from $X \to \mc B$ form a $\bb K$-vector space. In fact, if $\mc B$ is a Banach algebra, they form a $\bb K$-algebra.
\end{proposition}

\begin{proof}
    Clearly $kf$ is simple measurable since $\im(kf) = k\im(f)$, and $(kf)^{-1}(ky) = f^{-1}(y)$.

    Thus, if $f$ and $g$ are simple measurable, then let \[f = \sum_{i = 1}^m a_i\chi_{A_i}\quad \text{and}\quad g = \sum_{i = 1}^n b_i\chi_{B_i}.\] as in the previous lemma.

    For convinience, let $A_{m + 1} = X \setminus \bigcup A_i$, $B_{n + 1} = X \setminus \bigcup A_j$, and $a_{m + 1} = 0 = b_{m + 1}$. Then, let $C_{ij} = A_i \cap B_j$ and $c_{ij} = a_i + b_j$. We can write \[f + g = \sum_{i = 1}^{m + 1} \sum_{j = 1}^{n + 1} c_{ij}\chi_{C_{ij}}.\] 
    
    However, the $c_{ij}$ might be zero and the $C_{ij}$ might be empty, so we remove those terms from the summation. In addition, if there are two $c_{ij}$s that are equal, we combine those terms in the summation. After doing these steps, we have a summation in the form required by the previous lemma, and thus we are done.

    The proof for multiplication when $\mc B$ is an Banach algebra is similar.
\end{proof}

\begin{corollary}
    The simple measurable functions are the span of the indicator functions of measurable sets.
\end{corollary}

Finally, we define the integral for simple measurable functions using the rules we stated at the beginning of the chapter.

\begin{definition}
    For a simple measurable function $f \colon X \to \mc B$, we define the \textbf{integral} of $f$ to be \[\int f \dd{\mu} = \sum_{y \in \im(f) \setminus \qty{0}} \mu(f^{-1}(\{y\}))y.\]
\end{definition}

Clearly $\int \chi_E \dd{\mu} = \mu(E)$, and satisfies the two properties we proposed at the beginning of this section:

\begin{proposition}
    $\int \chi_E \dd{\mu} = \mu(E)$, and the integral is $\bb K$-linear.
\end{proposition}

\section{Measurable Functions}

Integrating simple functions isn't terribly interesting though, so we want to extend integration to functions that are non-simple. We can do this by considering the class of functions that are well approximable by SMFs, which turns out to be incredibly rich.

\begin{definition}
    If $f \colon X \to \mc B$, (where $\mc B$ could be just a pointed metric space), we say $f$ is \textbf{$\mc S$-measurable} if there exists a sequence $\qty{f_n}$ of SMFs that converges pointwise to $f$.

    If we have a measure $\mu$ on $\mc S$, the null sets form a $\sigma$-ring. In this situation, we say that $f$ is \textbf{$\mu$-measurable} if there exists a sequence $\qty{f_n}$ that converges to $f$ $\mu$-\emph{almost everywhere}, i.e., there exists a null set $N$ such that $f_n \to f$ on $X \setminus N$.
\end{definition}

To make the theory work smoothly, however, we need the following non-obvious fact:

\begin{proposition}
    If $f_n \to f$, where the functions $f_n$ are $\mc S$-measurable, then $f$ is measurable.
\end{proposition}

We will prove this later, but first we try and isolate properties of measurable functions to make this easier to prove.

\begin{proposition}
    If $\qty{f_n}$ are SMFs, and $f_n \to f$ pointwise, then \[\im(f) = \ol{\qty(\bigcup_{i = 1}^\infty \im(f_i))},\] and thus $\im(f)$ is separable, i.e. is contained in a closure of a countable set.
\end{proposition}

\begin{proposition}
    The carrier of a measurable function $f$ is contained in the union of the carriers of $f_n$, which are SMFs converging to $f$, and thus there is a set $E \in \mc S$ where $\carrier(f) \subseteq E$.
\end{proposition}

Now we get the main property
\begin{proposition}
    If $f$ is measurable then if $U$ is open in $\mc B$, we have $f^{-1}(U) \cap E \in \mc S$.
\end{proposition}

\begin{proof}
    $x \in f^{-1}(U)$ iff $f(x) \in U$. Now, we define $U_m$ to be the set $\qty{x \in U \mid d(x, U') > 1/m}$. Then, we have $\ol{U_m} \subseteq U_{m - 1}$. We can see that $f(x) \in U$ iff there exists $n$ such that $f(x) \in U_n$. Now, if $f_k \to f$ is any sequence of functions converging to $f$ pointwise, then using approximation and the triangle inequality, we see that $f(x) \in U$ if and only if there exists a $n$ and $K$ such for $k \ge K$ such that $f_k(x) \in U_n$.

    Now we have \[x \in f^{-1}(U) \cap E\iff x \in \bigcup_{n = 1}^\infty \bigcup_{K = 1}^\infty \bigcap_{k = K}^\infty f_k^{-1}(U_n) \cap E,\] thus $f^{-1}(U) \cap E \in \mc S$ for $f_k$ simple.
\end{proof}

\begin{corollary}
    If $f$ is measurable, for any Borel set $A \subseteq \mc B$, then $f^{-1}(A) \in \mc S$. \textbf{This is the traditional definition of measurable}.
\end{corollary}

\begin{corollary}
    If $f_n$ is a sequence of measurable functions converging to $f$ then $f$ satisfies
    \begin{enumerate}
        \item $\im(f)$ is separable.
        \item There is a set $E \in \mc S$ where $\carrier(f) \subseteq E$.
        \item If $f_n$ is a sequence of measurable functions and $f_n \to f$, then $f^{-1}(A) \cap E\in \mc S$ for all Borel sets $A \subseteq \mc B$. 
    \end{enumerate}
\end{corollary}

Now, to prove that limits of measurable functions are measurable, we just need to prove that functions that satisfy 1, 2, and 3 are measurable.

\begin{theorem}
    If $f \colon X \to \mc B$ has properties 1, 2, and 3, then it is measurable.
\end{theorem}

\begin{proof}
    In this proof, we will work implicity with $X = E$ to avoid many intersections. Now, to construct a sequence of SMFs that converge to $f$, we enumerate a sequence $\qty{b_j}$ that is dense in $\im(f)$. Set $C_{ji} = B_{1/j}(b_i)$, whose preimage is in $\mc S$, and we order the pairs $(j, i)$ lexicographically. Now, for fixed $n$, we construct $f_n$ as follows: disjointize the $C_{ji}$ for $j, i \le n$ in reverse of the order given. Thus, $E_{nn} = C_{nn}$, and $E_{ij} = C_{ji} \setminus \bigcup \qty{ C_{\ell k} \mid ji < \ell k \le nn, i \le n}$. Now we let $f_n = \sum_{ji \le nn, i \le n} b_i \chi_{E_{ji}}$.

    Now we prove that $f_n \to f$ pointwise. If we fix $x \in X$ and $\veps > 0$, there exists some point $b_N$ where $\norm{f(x) - b_N} < \veps$, and thus $x$ is contained within $C_{NK}$ for $K = \lceil 1/\veps \rceil$. Now, if we take $M = \max(N, K)$, then we can see for all $m \ge M$, $f_m(x)$ is either $b_N$, or some other $b$ such that $\norm{f(x) - b} < 1/M < \veps$.
\end{proof}