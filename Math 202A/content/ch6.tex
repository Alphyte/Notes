\chapter{Properties of Measures}

\section{Properties of Extensions}
Now, we prove a few properties of this constructed measure.

\begin{definition}
    If $\mc F$ is a heriditary family of subsets of $X$, and $\nu \colon \mc F \to [0, \infty]$, we say $\nu$ is \textbf{complete} if when $\nu(E) = 0$, for all $F \subseteq E$, we have $F \in \mc F$, and $\nu(F) = 0$.
\end{definition}
\begin{proposition}
    $\nu$ is complete.
\end{proposition}

\begin{proof}
    If $\nu(E) = 0$ and $B \subseteq E$, then for any set $A$, $\nu(A \cap B) = 0$, and $\nu(A \setminus B)$  must be equal to $\nu(A)$ due to monotonicity and subadditivity.
\end{proof}

Note that a similar proof also implies that any set with outer measure $0$ is measurable with measure $0$.

We may assume extensions are unique, but this does not hold:

\begin{proposition}
    Extensions are not necessarily unique.
\end{proposition}

\begin{proof}
    Consider $\bb R$ with the prering $\mc P$ consisting of all countable sets, and define $\mu(P) = 0$ for all $P \in \mc P$. Then when extending this to $S(\mc P)$, we can define the measure to be the zero measure or the measure that attains infinity on uncountable sets.
\end{proof}

Another example is given by:

\begin{example}
    Consider $\mc P$ to be the prering of half open intervals $[a, b)$ and define $\mu$ to be infinite on all of these.    
\end{example}

We notice that these all happen due to the extensions attaining infinite values, which motivates the following definition:

\begin{definition}
    If $(X, \mc P, \mu)$ is a premeasure space, then we say it is \textbf{$\sigma$-finite} if for all $E \in \mc P$ there exists a countable sequence of $E_j$ in $\mc P$ such that $E = \bigcup E_j$ and $\mu(E_j) < \infty$.
\end{definition}

\begin{theorem}
    If $(X, \mc P, \mu)$ is $\sigma$-finite then for all $\sigma$-rings $\mc S \subseteq M(\mu^*)$ containing $\mc P$, then $\mu^*\vert_\mc{S}$ is the unique measure on $\mc S$ extending $\mu$.
\end{theorem}

\begin{proof}
    If $\nu$ is any extension of $\mu$ to $\mc S$, if $G \in \mc S$ and $G \subseteq E \in \mc P$ 

    If $G$ is not contained in any $E$, then by the $\sigma$-finite condition, we can find a sequence of $E_j$ covering $G$.
\end{proof}

However, there is still a characterization of the measure induced by the outer measure even in the non $\sigma$-finite case.

\begin{proposition}
    If $(X, \mc P, \mu)$ is a premeasure, then for all $\sigma$-rings $\mc S \subseteq M(\mu^*)$ containing $\mc P$, then $\mu^*\vert_\mc{S}$ is the largest measure on $\mc S$ extending $\mu$.
\end{proposition}

\section{Continuity Properties of Measures}

We have the following ``continuity properties'' of measures:

\begin{proposition}
    If $E_j$ is an increasing sequence of sets, then $\mu(E_j) \nearrow \mu(\bigcup E_j)$.
\end{proposition}

\begin{proof}
    Disjointize the $E_j$ by setting $F_j = E_j \setminus E_{j - 1}$. Then we have $\mu(E_j) = \sum \mu(F_j)$, the latter of which is nondecreasing and converges upward to $\mu(\bigcup E_j)$.
\end{proof}

\begin{proposition}
    Similarly, if $E_j$ is a decreasing sequence of sets, then if the $E_j$ have finite measure, then $\mu(E_j) \searrow \mu(\bigcap E_j)$.
\end{proposition}

\begin{proof}
    Let $F_j = E_1 \setminus E_j$, then these sets are increasing and their union is $E_1 \setminus E := \bigcap E_j$. Thus we have that $\mu(E_1) - \mu(E_j) \to \mu(E_1) - \mu(E)$, implying that $\mu(E_j) \searrow \mu(E)$.
\end{proof}

\section{Translation Invariance of Measures}

The Lebesgue measure satisfies a property that we would intuitively expect.

\begin{proposition}
    The Lebesgue measure is translation-invariant.
\end{proposition}

\begin{proof}
    Problem Set.
\end{proof}

However, this property actually defines the Lebesgue measure:

\begin{definition}
    We call the $\sigma$-ring generated by the sets of the form $[a, b)$ the \textbf{Borel $\sigma$-ring}.
\end{definition}

\begin{proposition}
    Every measure $\mu$ on the Borel $\sigma$-ring that is translation-invariant and finite on the sets $[a, b)$ is a positive scalar multiple of the Lebesgue measure.
\end{proposition}

This actually has a remarkable generalization.

\begin{definition}
    If $(X, \mc T)$ is a topological space, we call the $\sigma$-ring generated by the sets in $\mc T$ the \textbf{Borel $\sigma$-ring}.

    However, when $X$ is locally compact (Hasudorff), the \textbf{Borel $\sigma$-ring} is instead defined as the $\sigma$-ring generated by compact subsets of $X$.
\end{definition}

\begin{theorem}[Haar]
    If $G$ is a locally compact group, then there is a nonzero Borel measure that is finite on compact sets which is invariant under left-translation.

    Moreover, any left-translation invariant Borel measure that is finite on compact sets is a nonnegative scalar multiple of the preceding measure.
\end{theorem}

\begin{theorem}[Weil]
    If $G$ is a group with $\sigma$-ring $\mc S$ and nonzero measure $\mu$, under certain hypothesis, if $\mc S$ and $\mu$ are left-translation invariant, then we can construct a topology on $G$ where it is a locally compact group and $\mu$ is its Haar measure.
\end{theorem}

On the problem set, we do a very special case where we construct the Haar measure on $\bb R/\bb Z$.

\section{Nonmeasurable Sets}

While the Borel $\sigma$-ring seems quite large, it is possible to find sets (using the Axiom of Choice) that are not Lebesgue-measurable. 

\begin{example}[Vitali Counterexample]
    Let $\sim$ be an equivalence relation on $[0, 1)$ with $x \sim y$ iff $y - x \in \bb Q$. Now, take $A$ to be the set containing a representative of each equivalence class.
\end{example}

\begin{proposition}
    $A$ is not measurable.
\end{proposition}

\begin{proof}
    Suppose $A$ was measurable. Then, $\mu(A + r) = \mu(A)$ for all $r \in \bb R$, due to translation invariance of the Lebesgue measure. We clearly have that $[0, 1)$ is the disjoint union of \[((A \cap [0, r)) + r) \sqcup ((A \cap (r, 1)) - 1 + r)\] for $r \in \bb Q \cap [0, 1)$.

    Now, if $\mu(A) = 0$, that would imply that $\mu([0, 1))= 0$, and if $\mu(A) > 0$, that would imply that $\mu([0, 1)) = \infty$. Thus $A$ cannot be measurable.
\end{proof}

This counterexample can also be constructed on the circle.