\section*{Administrative Matters}
Midterm date options:
\begin{enumerate}
    \item Fri Oct 14
    \item Mon Oct 17
    \item Wed Oct 19
\end{enumerate}

Weekly psets due Friday midnight.

\chapter{General Topology}

\section{Lecture 1: Metrics}

\begin{definition}[Metric Space]
    A metric on a set $X$ is a function $d\colon X^2 \to \bb R_{\ge 0}$ such that
    \begin{enumerate}
        \item $d(x, y) = 0$ if and only if $x = y$.
        \item $d(x, y) = d(y, x)$.
        \item $d(x, z) \le d(x, y) + d(y, z)$.
    \end{enumerate}

    $(X, d)$ is then called a metric space.
\end{definition}

If condition 1 is omitted and replaced only with the condition that $d(x, x) = 0$, then $d$ is called a semimetric (or quasimetric). If distances are allowed to be infinite, then $d$ is called an extended metric.

If $(X, d)$ is a semimetric space, then if we define a relation $\sim$ such that $x \sim y$ iff $d(x, y) = 0$, it is an equivalence relation. In this case, $d$ ``drops'' to a metric on the equivalence classes $X/\!\sim$. To show this, we just need to prove that $\tilde d([x], [y]) = d(x, y)$ is a well-defined function, which utilizes the triangle inequality.

\begin{example}
    If $G$ is a connected weighted graph, then we can define a metric on $G$ by defining the distance between two vertices to be the length of the minimal path connecting them.

    If $G$ is unconnected, by assigning the distance of infinity between points in separate components, we define a extended metric.
\end{example}

\begin{definition}[Norm]
    If $V$ is a vector space then $\norm{\cdot}\colon V \to \bb R_{\ge 0}$ is a norm on $V$ if 
    \begin{enumerate}
        \item $\norm{v} = 0$ iff $v = 0$
        \item $\norm{\lambda v} = \abs{\lambda}\norm{v}$
        \item $\norm{v + w} \le \norm{v} + \norm{w}$
    \end{enumerate}
\end{definition}

A norm on $V$ clearly induces a metric on $V$ by defining $d(v, w) = \norm{v - w}$. 

\begin{example}[Euclidean metric]
    The real line $\bb R$ and the euclidean spaces $\bb R^n$ with the euclidean metric $d(x, y) = \norm{x - y}_2$.
\end{example}

\begin{example}[$p$-norms]
    The $p$-norms $\norm{(x_1, \dots, x_n)}_p = \qty(\sum x_i^p)^{1/p}$ induce a metric on $\bb R^n$.
\end{example}

\begin{definition}[Restricted metric]
    If $(X, d)$ is a metric space and $Y$ is a subset of $X$, then we can restrict $d$ to $Y$ and $d\vert_Y$ is a metric on $Y$.
\end{definition}

It is common to consider the restriction of the Euclidean metric onto subsets of Euclidean space.

\begin{example}[$L^p$ norms]
    If $V = \mc C^0([0, 1])$, then examples of norms on $V$ include 
    \begin{enumerate}
        \item $\norm{f}_\infty = \sup{\abs{f(t)}}$.
        \item $\norm{f}_1 = \int_0^1 \abs{f(t)} \dd{t}$
        \item $\norm{f}_2 = \qty(\int_0^1 \abs{f(t)}^2 \dd{t})^{1/2}$
        \item $\norm{f}_p = \qty(\int_0^1 \abs{f(t)}^p \dd{t})^{1/p}$ for $1 \le p < \infty$.\footnote{The $\infty$-norm can be viewed as the limit of the $p$-norms as $p \to \infty$.}
    \end{enumerate}
\end{example}

This generalizes to bounded closed regions of $\bb R^n$, where all continuous functions are integrable (due to compactness and the extreme value theorem). The proof that these are actually norms that satisfy the triangle inequality will come later (see the Minkowski and H\"older inequalities).

\section{Lectures 2, 3: Topology in Metric Spaces}

\subsubsection*{Categories}

Categories contain objects and morphisms, that satisfy the following properties:
\begin{enumerate}
    \item There is a morphism from any object $A$ to itself, called the identity $1_A$.
    \item Morphisms can be composed.
\end{enumerate}

Metric spaces can form a category under a few choices of collections of morphisms:
\begin{itemize}
    \item Isometries: $i\colon (X, d_X) \to (Y, d_Y)$ such that $d_X(x, y) = d_Y(i(x), i(y))$ for all $x, y$.
    
    Surjective isometries form a subcategory of the above category, ,where every map is an isomorphism.
    \item Lipschitz maps: $f\colon (X, d_X) \to (Y, d_Y)$, where there exists a constant $C$ (which may depend on $f$), such that $d_Y(f(x), f(y)) \le Cd_X(x, y)$.
    
    We can also form the subcategory of bi-Lipschitz isomorphisms.
    
    \item Uniformly continuous maps: $f\colon (X, d_Y) \to (Y, d_Y)$ such that for all $\varepsilon > 0$ there exists $\delta > 0$ such that when $d_X(x, y) \le \delta$, then $d_Y(f(x), f(y)) < \varepsilon$.
    \item Continuous maps: $f\colon (X, d_Y) \to (Y, d_Y)$ such that for all $x \in X$, for all $\varepsilon > 0$ there exists $\delta > 0$ such that when $d_X(x, y) \le \delta$, then $d_Y(f(x), f(y)) < \varepsilon$.
\end{itemize}

These conditions get weaker as we go down, and continuity can be generalized to non-metric spaces which gives rise to general topology. The study of uniformly continuous maps is much less common, but gives rise to uniformities.

\begin{definition}[Convergence]
    If $\qty{x_n}$ is a sequence in a metric space $(X, d)$, then $x_n \to x$ if $\forall \veps > 0,\, \exists N$ such that $n > N$ implies that $d(x_n, x)$.
\end{definition}

\begin{definition}[Cauchy sequence]
    If $\qty{x_n}$ is a sequence in a metric space $(X, d)$, then $x_n \to x$ if $\forall \veps > 0,\, \exists N$ such that $n, m > N$ implies that $d(x_n, x_m)$.
\end{definition}

\begin{definition}[Completeness]
    $(X, d)$ is complete if every cauchy sequence converges to a point.
\end{definition}

For any metric space that isn't complete, we want to form a ``completion" of it.

\begin{definition}[Density]
   If $S \subseteq X$, $S$ is dense if for all $x$ in $X$ and all $\veps < 0$ there exists $s \in S$ where $d(s, x) < \veps$.
\end{definition}

\begin{definition}[Completion]
    A completion of a metric space $(X, d)$ is a metric space $(\overline X, \overline d)$ with an inclusion isometry $i \colon X \to \overline{X}$ such that the image of $i$ is dense in $\overline{X}$. In addition, $\overline{X}$ is a complete metric space.
\end{definition}

One way to define a completion for any metric space is by defining a semimetric $\tilde d$ on the set of Cauchy sequences $\mathrm{cs}(X, d)$, where $\tilde d\qty(\qty{x_n}, \qty{y_n}) = \lim_{n \to \infty} d(x_n, y_n)$, then by dropping that to a metric on the equivalence classes induced by the semimetric.

We can just define $f \colon X \to \overline X$ as $f(x) = [\qty{x, x, x, \dots}]$. To show that $\overline X$ is complete, we consider Cauchy sequences of Cauchy sequences, approximate their elements by elements in $X$, and show that the sequence converges to the sequence generated by the approximation.

\begin{example}
    If $f \in V = \mc C^0([0, 1])$, the norm $\norm{f}_\infty = \sup(\abs{f(x)})$ induces a complete metric space (due to the limit of uniformly continous functions converging to a continuous function).

    However, in the $L^1$ norm $\norm{f}_1 = \int_0^1 \abs{f(x)} \dd{x}$, $V$ is not complete (you can construct a sequence of functions that converges to a step function).
\end{example}

Now to answer the question ``What is the completion of $V$ under the $L^p$ norms", we must look to measure theory (2nd half of this course).

If $(X, d_X)$ and $(Y, d_Y)$ are metric spaces, if $f \colon X \to Y$, recall that if $x \in X$, for all $\varepsilon > 0$ there exists $\delta > 0$ such that when $d_X(x, y) \le \delta$, then $d_Y(f(x), f(y)) < \varepsilon$, then $f$ is continuous.

\begin{definition}[Open ball]
    The open ball of radius $r$, $B_r(x) = \qty{y \in X \mid d_X(x, y) < r}$.
\end{definition}

This allows us to refine our definition of continuity:
\begin{definition}[Continuity]
    If $f \colon X \to Y$, then $f$ is continuous if for all $x$, for all $\veps > 0$, there exists $\delta > 0$ such that $f(B_\delta(x)) \subseteq B_\veps(f(x))$.
\end{definition}

\begin{definition}[Open set]
    The set $A \subseteq X$ is open if for all $a \in A$, there exists an open ball around $a$ that is contained within $a$.
\end{definition}

\begin{proposition}
    Open balls are open.
\end{proposition}

\begin{proof}
    If $a \in B_r(x)$, then $B_{r - d_X(a, x)}(a)$ is contained within $B_r(x)$ due to the triangle inequality.
\end{proof}

\begin{proposition}
    If $f$ is continuous, then if $U$ is open in $Y$ such that $x \in X$, then there is an open ball around $x$ that is contained within the preimage of $U$.
\end{proposition}

A direct corollary of this proposition is 
\begin{corollary}
    $f\colon X \to Y$ is continuous is equivalent to the following condition: If $U$ is open in $Y$, then $f^{-1}(U)$ is open in $X$.
\end{corollary}

Now, we consider properties of the set $\mc T$ consisting of open sets in $X$:

\begin{theorem}
    \begin{enumerate}
        \item $\emptyset, X \in \mc T$.
        \item Arbitrary unions of sets in $\mc T$ are also in $\mc T$.
        \item Finite intersections of sets in $\mc T$ are in $\mc T$.
    \end{enumerate}
\end{theorem}
\begin{proof}
    Trivial.
\end{proof}

These properties give rise to the definition of a topology.

\begin{definition}[Topology]
    A topology on a set $X$ is a collection $\mc T \subseteq 2^X$ such that \begin{enumerate}
        \item $\emptyset, X \in \mc T$.
        \item Arbitrary unions of sets in $\mc T$ are also in $\mc T$.
        \item Finite intersections of sets in $\mc T$ are in $\mc T$.
    \end{enumerate}

    The sets in $\mc T$ are called open sets.
\end{definition}

\section{General Topology}

We recall the definition of a topology:

\begin{definition}[Topology]
    A topology on a set $X$ is a collection $\mc T \subseteq 2^X$ such that \begin{enumerate}
        \item $\emptyset, X \in \mc T$.
        \item Arbitrary unions of sets in $\mc T$ are also in $\mc T$.
        \item Finite intersections of sets in $\mc T$ are in $\mc T$.
    \end{enumerate}

    The sets in $\mc T$ are called open sets.
\end{definition}

We then extend the definition of continuity to general topologies.
\begin{definition}[Continuity]
    If $(X, \mc T_X)$ and $(Y, \mc T_Y)$ are topological spaces, then $f$ is continuous if for all $U \in \mc T_Y$, $f^{-1}(U) \in \mc T_X$.
\end{definition}

Topologies on a set $X$ form a poset under the subset relation.

\section{Bases and Subbases}

\begin{proposition}
    If $X$ is a set, then the arbitrary intersection of topologies is a topology on $X$.
\end{proposition}

\begin{proof}
    These details are routinely verified.
\end{proof}

\begin{corollary}
    Given any collection $\mc S$ of subsets of $X$, there is a smallest topology containing $\mc S$.
\end{corollary}

\begin{proof}
    This topology is the intersection of all the topologies that contain $\mc S$. We say that $\mc S$ generates this topology.
\end{proof}

\begin{definition}[Subbase]
    Let $\mc T$ be a topology on $X$, and let $\mc S \subseteq 2^X$, with $X = \bigcup \mc S$. We say that $\mc S$ is a subbase of $\mc T$ if $\mc T$ is the topology generated by $\mc S$.
\end{definition}

Subbases can be useful for studying topologies in the same way that a basis is useful for studying vector spaces. For example, see \autoref{prop:subbasecontinuity}.

We can explicitly characterize the topology generated by a set $\mc S$. We first state a lemma:

\begin{lemma}
    If $\mc I \subseteq 2^X$ be closed under finite intersections and $\bigcup \mc I = X$, then the topology generated by $\mc I$ consists of the arbitrary unions of sets in $\mc I$ (with union over $0$ sets being $\emptyset$).
\end{lemma}

\begin{proof}
    The only part that is not immediately obvious is the closure under finite intersections. If $\mc O_1 \bigcup_A U_\alpha$ and $\mc O_2 = \bigcup_B V_\beta$, then \[ \mc O_1 \cap \mc O_2 = \bigcup_{A, B} U_\alpha \cap V_\beta.\]
\end{proof}

\begin{proposition}
    If $\mc S \subseteq 2^X$ whose union is $X$, then the topology $\mc T$ generated by $\mc S$ consist of the arbitrary unions of $\mc I^{\mc S}$ which are defined as the set of finite intersections of sets in $\mc S$.
\end{proposition}

\begin{definition}[Base]
    A collection $\mc B \subseteq 2^X$ is called a base for a topology $\mc T$ on $X$ if $\mc T$ consists of the arbitrary unions of elements in $\mc B$.
\end{definition}

\begin{example}
    \begin{enumerate}
        \item The collection of finite intersections $\mc I^{\mc S}$ of a subbase $\mc S$.
        \item The set of open balls in a metric space.
        \item The set of intervals of the form $(-\infty, a)$ and $(b, \infty)$ in $\bb R$ under the standard topology.
    \end{enumerate}
\end{example}

Notice how the second and third examples are not closed under finite intersections. 

\begin{proposition}
    For any base $\mc B$, any finite intersection of sets in $\mc B$ must be a union of elements in $\mc B$.
\end{proposition}

\begin{proposition}
    \label{prop:subbasecontinuity}
    If $(X, \mc T_X)$ and $(Y, \mc T_Y)$ are topological spaces, and let $f \colon X \to Y$. If $\mc S_Y$ is a subbase of $\mc T_Y$, then $f$ is continuous iff for all $U \in \mc S_Y$, we have $f^{-1}(U)$ is open.
\end{proposition}

\begin{proof}
    First show that the preimages of sets in $\mc I_{\mc S_Y}$ are open, and then show that the preimages of all sets in $\mc T_Y$ are open.
\end{proof}

\begin{definition}[Initial Topologies]
    If $X$ is a set, $(Y_\alpha, \mc T_\alpha)$ are topological spaces, and for each $\alpha$, $f_\alpha\colon X \to Y_\alpha$. Then, the initial topology is the smallest topology on $X$ that makes all $f_\alpha$ continuous.
\end{definition}

This initial topology is just the topology generated from the preimages of open sets in $Y_\alpha$ under $f_\alpha$.

\begin{example}
    If $X \subseteq Y$ and $(Y, \mc T_Y)$ is a topological space, and if $j$ is the inclusion map from $X$ to $Y$, then the initial topology generated by $j$ consists of the intersection of open sets with $X$, which is also called the relative (or subspace) topology. 
\end{example}