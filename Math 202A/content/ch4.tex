\chapter{Compactness}

\begin{definition}[Open Cover, Subcover]
    If $X$ is a topological space, by an open cover we mean a collection $\mathcal C = \qty{U_\alpha}$ of open sets such that $\bigcup \mathcal C = X$.

    By a subcover of $\mc C$, we mean a collection $\mc D \subseteq \mc C$ where $\bigcup \mc D = X$.
\end{definition}

\begin{definition}[Compactness]
    $X$ is compact if and only if every open cover has a finite subcover. If $A \subseteq X$, then $A$ is compact if $A$ is compact in its relative topology.
\end{definition}

\begin{proposition}
    $A$ is compact iff for any collection $\mc E$ of open sets such that $\bigcup \mc E \supseteq A$, there is a finite subset $\mc S \subseteq \mc E$ that also covers $A$, that is, $\bigcup \mc S \supseteq A$. 
\end{proposition}

\begin{proposition}
    Any closed subset of a compact space is compact.
\end{proposition}

\begin{proof}
    If $\mc E$ is covers $A \subseteq X$, then $\mc E \bigcup \qty{A'}$ covers $X$, so we find a finite subcover and take out $A'$.
\end{proof}

Note that the converse is not true: Any subset of a space with the indiscrete topology is compact. However, when $X$ is Hausdorff, then this holds:

\begin{proposition}
    If $X$ is Hausdorff, then if $A \subseteq X$ is compact, then for all $x \notin A$, then there exist disjoint open sets $U$ and $V$ such that $A \in U$ and $x \in V$.
\end{proposition}

\begin{proof}
    If $x$ is in the complement of $A$, for all $y \in A$ we can find disjoint open sets $U_y, V_y$ with $x\in U_y$ and $y \in V_y$. Then $V_y$ covers $A$, and we can find a finite subcover $V_{y_k}$, $1 \le k \le n$. Then, we have $\bigcap U_{y_k}$ is an open set that contains $X$ and does not intersect $\bigcup V_{y_k}$, which is an open set containing $x$.
\end{proof}

We get two immediate corollaries:

\begin{corollary}
    \label{prop:compsuboft2closed}
    Any compact subset of a Hausdorff space is closed.
\end{corollary}

\begin{corollary}
    Any compact Hausdorff space is regular ($T_3$), i.e. for any closed set $A \subset X$, and any $x \notin A$, we have disjoint open sets $U, V$ such that $A \subset U$ and $x \in V$.
\end{corollary}

We can prove a slightly stronger statement: 

\begin{corollary}
    Any compact Hausdorff space is normal ($T_4$).
\end{corollary}

\begin{proof}
    If $A$ and $B$ are disjoint closed sets in a compact Hausdorff space $X$, then for any $y \in B$ there are disjoint open sets $U_y, V_y$ such that $A \subseteq V_y$ and $y \in U_y$, and we can find a finite subcover $U_{y_k}$ of the open cover $U_y$ of $B$, Now, we let $U = \bigcup U_{y_k}$ and $V = \bigcap V_{y_k}$ and we have $U$ disjoint from $V$, where $B \subseteq U$ and $A \subseteq V$. 
\end{proof}

Now, we examine maps on compact spaces.

\begin{proposition}
    \label{prop:imcompiscomp}
    If $X$ is compact and $f \colon X \to Y$, then $f(X)$ is compact.
\end{proposition}

\begin{proof}
    WLOG assume that $Y = f(X)$. If $\mc C$ is an open cover of $f(Y)$, then if $\mc C'$ is the set of preimages of sets in $\mc C$, then $\mc C'$ is an open cover of $X$, which we can take a finite subcover $\mc S' = \qty{f^{-1}(O_1), \dots, f^{-1}(O_n)}$, then we have that $\mc S = \qty{O_1, \dots, O_n}$ is a finite subcover of $Y$. 
\end{proof}

\begin{corollary}[Extreme Value Theorem]
    If $f \colon \bb R \to \bb R$ is continuous on an interval $[a, b]$, then $f$ has a maximum on $[a, b]$ and attains it on $[a, b]$. 
\end{corollary}

\begin{proposition}
    If $f\colon X \to Y$ for $X$ compact and $Y$ Hausdorff, then if $f$ is bijective, then it is a homeomorphism.
\end{proposition}

\begin{proof}
    Images of closed sets are closed due to \autoref{prop:imcompiscomp} and \autoref{prop:compsuboft2closed}.
\end{proof}

We can give an alternative characterization of compactness using closed sets.

\begin{proposition}
    $X$ is compact iff for any collection of closed sets has the finite intersection property, the intersection is nonempty.
\end{proposition}

\section{Tychonoff's Theorem}

We first state Tychonoff's theorem:

\begin{theorem}[Tychonoff's Theorem]
    \label{thm:tychonoff}
    If $\qty{X_\alpha}$ are a collection of compact spaces, then the space $\prod_\alpha X_\alpha$ is compact.
\end{theorem}

It is clear that this requires the Axiom of Choice, as $\prod_\alpha X_\alpha$ may be empty without it. It is also interesting that Tychonoff's Theorem implies AC, although Tychonoff's Theorem for Hausdorff compact spaces does not imply AC.

The proof of this theorem also requires AC, in the form of 

\begin{theorem}[Zorn's Lemma]
    \label{thm:zorn}
    If $P$ is a poset, then if any chain $C$ (totally ordered subset) in $P$ has an upper bound in $P$, then $P$ has a maximal element. (We call $P$ inductively ordered).
\end{theorem}

\nameref{thm:zorn} has wide applications across many fields, such as in the proof that every vector space has a basis, or that every proper ideal is contained in a maximal ideal.

\begin{definition}
    For any set $X$, a filter on $X$ is a collection $\mc F$ of nonempty subsets that is closed under intersections, and where all supersets of a set in the filter is also in the filter.

    An ultrafilter is a maximal filter.
\end{definition}

\begin{proof}[Proof of \nameref{thm:tychonoff}]
    If $\alpha \in \mc A$ indexes a family of compact spaces $X_\alpha$, then if $X = \prod X_\alpha$, then let $\mc C$ be a collection of closed subsets of $X$ with the finite intersection property. If $W_{\mc C}$ is the collection of families $\mc D$ of subsets of $X$ such that $\mc C \subseteq \mc D$ and such that $\mc D$ has FIP.

    Now, we show that $W_{\mc C}$ is inductively ordered: If $\Phi$ is a chain in $W_C$, then if $ \mc D_\Phi = \bigcup \Phi$, then let $A_1, \dots, A_n \in \mc D_\phi$, then $A_i \in \mc D_{\alpha_i}$, which we can just find the maximal index out of the $\alpha_i$, which we call $\alpha_0$. Then, $A_i \in \alpha_0$ for all $i$, which proves that $\bigcap A_i$ has nonempty intersection.

    We can then use Zorn's Lemma to obtain a maximal element $\mc D^*$ in $W_{\mc C}$. Now, we can see that $\mc D^*$ is closed under finite intersections because it is maximal. In addition, we also have that if $B \cap A \ne \emptyset$ for all $A \in \mc D^*$, then $B \in \mc D^*$, again due to maximality (we can use the previous property to prove this statement).

    Now, for any $\alpha \in \mc A$, we have the continuous projection $\pi_\alpha\colon X \to X_\alpha$, now we claim that for any $\mc D \in W_{\mc C}$, the set of $\qty{\pi_\alpha(A) \mid A \in \mc D}$ has FIP since if $x \in \bigcap A_j$, then $\pi_\alpha(x) \in \bigcap \pi_\alpha(A_j)$. We also trivially have that the closures $\ol{\pi_\alpha(A)}$ has the FIP. Now, applying to $\mc D^*$, we have that $\bigcap_{A \in \mc D^*} \ol{\pi_\alpha(A)}$ is nonempty since $X_\alpha$ is compact.

    Now, applying AC, we can pick $x \in X$ where $\pi_\alpha(x) \in \bigcap_{A \in \mc D^*} \ol{\pi_\alpha(A)}$. Now, we prove that $x \in \ol A$ for all $A \in \mc D^*$, using the subbase of the product topology. If we let $U = \pi_\alpha^{-1}(U_\alpha)$, then $x \in U$ implies that $U \cap A \ne 0$ for all $A \in \mc D^*$, since we have $x_\alpha \in \ol{\pi_\alpha(A)}$ and $U_\alpha$, then $\pi_\alpha(A) \cap U_\alpha$ must be nonempty otherwise $\ol{\pi_\alpha(A)}$ would not contain $x_\alpha$. This implies that $\pi^{-1}(U_\alpha)$ has nonempty intersection with any set in $\mc D^*$, and thus belongs to $\mc D^*$.

    Now, by the first property of $\mc D^*$, then we have that finite intersections $\bigcap \pi_\alpha^{-1}(U_{\alpha_i})$ are in $\mc D^*$, thus $\mc D^*$ contains a neighborhood base of $\mc D^*$. Now, if $x \in U$ open in $X$, then there exists an open set $V$ in $\mc D^*$. Thus, for all $A \in \mc D^*$, we have that $x \in \ol A$. Finally, this results in the fact that $x$ is contained in $\bigcap \mc C$.
\end{proof}

This theorem has applications in 202B, such as the Aloglu theorem (unit ball in weak-* topology is compact), and when considering locally compact spaces and $\beta$-compactifications.

Now, we consider the following interesting result 

\begin{theorem}[Kelley]
    Tychonoff's Theorem is equivalent to AC.
\end{theorem}

\begin{proof}
    The reverse implication was proven above, and now if we assume Tychonoff's theorem, we need to show that $\prod X_\alpha$ is nonempty. We can take $\omega = \bigcup X_\alpha$ which is not in $\bigcup X_\alpha$ due to regularity. Now, if $Y_\alpha = X_\alpha \cup \qty{\omega}$, we can take $\mc T_\alpha = \qty{Y_\alpha, \emptyset, T_\alpha, \qty{\omega}}$, which is compact since there are finitely many open sets. The product $Y = \prod Y_\alpha$ is compact due to Tychonoff's theorem. If $F_\alpha = \pi_\alpha^{-1}(X_\alpha)$, then $F_\alpha$ is closed, and has the FIP, since for finite intersections $\bigcap F_{\alpha_i}$, we can consider $x_{\alpha_i} = F_{\alpha_i}$, and $x_\alpha = \omega$ otherwise. Thus due to compactness we have $\bigcap F_\alpha = \prod X_\alpha \ne \emptyset$, which proves the equivalence.
\end{proof}

\section{Totally Bounded Sets}

\begin{definition}
    We say a metric space $(X, d)$ is \textbf{totally bounded} if for all $\veps > 0$, there exists a finite set $\qty{x_i}$ such that \[X = \bigcup_{i = 1}^n B_\veps(x_i).\]
\end{definition}

It is trivial that compactness implies totally bounded. We now prove a partial converse:

\begin{theorem}
    Complete and totally bounded metric spaces are compact.
\end{theorem}

\begin{proof}
    If $X$ is totally bounded, then if $\qty{x_n}$ is a sequence in $X$, we have for any $\veps > 0$ that $X = \bigcup B_\veps(x_i)$. Due to finiteness, for at least one $x_i$, we have infinitely many points in the sequence in $B_\veps(x_i)$. Taking $\veps = 1$, we have $A_1 = B_1(y_1)$ which has infinitely many points of $x_n$ inside, which means we can choose $A_{1/2} = B_{1/2}(y_2) \cap A_1$ with infinitely many points of $x_n$ inside, and we can repeat this process to obtain a sequence $A_1 \supseteq A_{1/2} \supseteq A_{1/4} \supseteq \cdots$ with successively decreasing diameters. By completeness, there exists an (unique) point in the intersection, and thus we have found an accumulation point.
\end{proof}

\begin{corollary}[Heine-Borel Theorem]
    A subset of Euclidean space is compact iff it is closed and bounded.
\end{corollary}

We can now study compact sets in $\mathrm{BC}(X, M)$ with the $d_\infty$ metric, where $X$ is a topological space and $M$ is a complete metric space. These correspond precisely to the totally bounded sets.

\begin{definition}
    If $X$ is a topological space and $(M, d)$ is a metric space, then a family of functions $\mc F \subseteq \mathrm C(X, M)$ is \textbf{equicontinuous at a point $x$} if for any $\veps > 0$, there exists an open neighborhood $\mc O$ such that $d(f(y), f(x)) < \veps$ whenever $y \in \mc O$.

    We say $\mc F$ is \textbf{equicontinuous} if it is equicontinuous at all points in $X$.
\end{definition}

\begin{theorem}[Arzela-Ascoli]
    Let $X$ and $M$ be as above, and let $X$ be compact. If $\mc F \subseteq \mathrm{BC}(X, M)$ with $\mc F$ pointwise totally bounded, and equicontinuous, then $\mc F$ is totally bounded.
\end{theorem}

\begin{proof}
    If $\veps > 0$ is given, for each $x$ there is an open set $O_x$ such that $y \in O_x$, then $d(f(y), f(x)) < \veps/4$. By compactness, there are $x_1, \dots, x_n$ such that \[X = \bigcup_{i = 1}^n O_{x_i}.\] Now, for each $j$, the set $I_j = \qty{f(x_j) \mid f \in \mc F}$ is totally bounded, so if we choose $S_j$ to be a finite subset of $I_j$ that is $\veps/4$-dense in $I_j$. We can then take $S = \bigcup S_j$. Now, if we let $\Psi= \qty{\psi \colon \qty{1, \dots, n} \to S}$, which is finite, then we can define $\mc F_\psi = \qty{f \in \mc F \mid \forall j.\ d\qty(f\qty(x_j), \psi(j)) < \veps/4}$. We thus have $\mc F = \bigcup \mc F_{\psi}$.

    Now, we prove that $\operatorname{diam}\qty(\mc F_\psi) < \veps$. For all $y$, we have that $y \in O_{x_i}$ for some $i$. Thus, if $f, g \in \mc F_\psi$, then \[d(f(y), g(y)) \le d(f(y), f(x_i)) + d(g(y), g(x_i)) + d(f(x_i), \psi(x_i)) + d(g(x_i), \psi(x_i)) < 4(\veps/4) = \veps.\] Then, $\mc F_\psi$ is contained within $B_\veps(f_\psi)$ for any $f_\psi \in \mc F_\psi$, and so $X = \bigcup B_\veps(f_\psi)$.
\end{proof}

\section{Locally Compact Spaces}

\begin{definition}
    A topological space $X$ is \textbf{locally compact} if for all $x \in X$, there exists a open set $x \in O$ such that $\ol O$ is compact.
\end{definition}

For now, we assume that $X$ is Hausdorff for simplicity.

\begin{proposition}
    If $X$ is locally compact, then for any point and open set $x \in O$, there exists an open set $x \in U \subseteq \ol U \subseteq O$.
\end{proposition}

\begin{proof}
    Let $O' = O \cap O_1$, where $\ol{O_1}$ is compact. Now, due to regularity of $\ol{O'}$ which is also compact, and disjoint open sets $U$ and $V$ such that $x \in U$ and $\partial O \subseteq V$. Then $\ol U$ does not contain any points in $\partial O$ and thus $\ol U \subseteq O' \subseteq O$. Furthermore, $\ol U$ is compact due to being a closed subset of $\ol{O'}$.
\end{proof}

A stronger proposition is the following:

\begin{proposition}
    If $X$ is locally compact, then for any compact subset of an open set $C \subseteq O$, there exists an open set $C \subseteq U \subseteq \ol U \subseteq O$.
\end{proposition}

\begin{proof}
    This just follows by taking points in $C$ and applying the previous proposition, then using compactness.
\end{proof}

\begin{definition}
    For $f \in \mathrm C(X, V)$ where $V$ is a vector space and $X$ is locally compact, the \textbf{support} of $f$, $\supp(f)$ is defined as the closure of the set $\qty{x \colon f(x) \ne 0}$.
\end{definition}

\begin{definition}
    We say that $f$ has \textbf{compact support} if $\supp(f)$ is compact. We denote the set of continuous compactly-supported functions by $\mathrm C_{\mathrm c}(X, V)$. 
\end{definition}

\begin{proposition}
    For $X$ locally compact and $C$ compact, $O$ open, then there is a function $f \in \mathrm C_{\mathrm c}(X, \bb R)$ where $f(C) = 1$ and $f(O') = 0$.
\end{proposition}

\begin{proof}
    Find $U$ such that $C \subseteq U \subseteq \ol U \subseteq O$ and apply Urysohn's Lemma.
\end{proof}

\begin{definition}
    $\mathrm{C}_\mathrm{b}(X, \bb R)$ denotes the set of all bounded functions $X \to \bb R$. Using the $L^\infty$ norm, we have that the closure of $\mathrm C_{\mathrm c}(X, \bb R)$ is $\mathrm C_\infty(X, \bb R)$, which are the set of functions that vanish at infinity.
\end{definition}

\section{A Quick C*-algebra Interlude}

We start with the following proposition:

\begin{proposition}
    If $X$ is a compact Hausdorff space, then the only unital homomorphisms between $\mrm C(X)$ and $\bb K$ are the evaluation homomorphisms $\varphi_x(f) = f(x)$.
\end{proposition}

If $A$ is any $\bb K$-algebra, then by $\widehat A$ we mean the set of unital homomorphisms $A \to \bb K$. If $a \in A$, we define $\widehat a$ to be the function $\widehat a \colon \hat A \to \bb K$ to be $\widehat a(\varphi) = \varphi(a)$. We give $\wh A$ the weakest topology that makes these functions continuous, that is, the initial topology generated by the $\wh a$.

\begin{proposition}
    For $A = \mrm C(X)$, we have that $x \mapsto \varphi_x$ is a homeomorphism from $X$ to $\widehat A$.
\end{proposition}

If $K = \bb C$, then $A = \mathrm C(X)$ has an involution, where $f^*(x) = \ol{f(x)}$. It is obvious that $\norm{f^*f}_\infty = \norm{f}_\infty^2$. This seemingly innocent property actually characterizes $C(X)$ with the $\infty$-norm:

\begin{theorem}[Little Gelfand-Naimark theorem, 1943]
    If $A$ is a commutative Banach algebra with involution. If $\norm{a^*a} \norm{a}^2$ for all $a \in A$, then $A$ is isomorphic (isometric *-isomorphism) to $C(\wh A)$.
\end{theorem}

Now if we consider $X$ locally compact, then we let $A = \mrm C_\infty(X)$. Similarly to the cases above, $\mrm C_\infty(X) \cong \mrm C_\infty(\wh A)$ and $X \cong \wh A$.

\begin{definition}
    A \textbf{Hilbert space} is a inner product space that is complete under its induced norm.

    We write $\mrm B(\mc H)$ to be the set of all continuous operators on $\mc H$.
\end{definition}

$\mrm B(\mc H)$ has an involution, the adjoint, and $\norm{T^*T} = \norm{T}^2$. However, it isn't commutative, but this motivates the general form of the Gelfand-Naimark theorem, where any space satisfying this property is isomorphic in a sense to $\mrm B(\mc H)$ for some Hilbert space $\mc H$.

Algebras that satisfy this property are called $C^*$-algebras.