\chapter{Topological Spaces}

We recall the definition of a topology:

\begin{definition}[Topology]
    A topology on a set $X$ is a collection $\mc T \subseteq 2^X$ such that \begin{enumerate}
        \item $\emptyset, X \in \mc T$.
        \item Arbitrary unions of sets in $\mc T$ are also in $\mc T$.
        \item Finite intersections of sets in $\mc T$ are in $\mc T$.
    \end{enumerate}

    The sets in $\mc T$ are called open sets.
\end{definition}

We then extend the definition of continuity to general topologies.
\begin{definition}[Continuity]
    If $(X, \mc T_X)$ and $(Y, \mc T_Y)$ are topological spaces, then $f$ is continuous if for all $U \in \mc T_Y$, $f^{-1}(U) \in \mc T_X$.
\end{definition}

Topologies on a set $X$ form a poset under the subset relation.

\begin{definition}[Closed sets]
    The closed sets in $(X, \mc T)$ are precisely the complement of open sets in $X$.
\end{definition}

\begin{proposition}[Properties of closed sets]
    \begin{enumerate}
        \item $\emptyset, X$ are closed.
        \item Arbitrary intersections of closed sets are also closed.
        \item Finite unions of closed sets are closed.
    \end{enumerate}
\end{proposition}

\begin{definition}[Closure]
    The closure $\overline A$ of an arbitrary set $A$ is the smallest closed set containing it.
\end{definition}

\begin{definition}[Density]
    Given $A \subseteq B$, we say $A$ is dense in $B$ if $B \subseteq \overline{A}$. Equivalently, $\overline{A} = \overline{B}$.
\end{definition}

\section{Bases and Subbases}

\begin{proposition}
    If $X$ is a set, then the arbitrary intersection of topologies is a topology on $X$.
\end{proposition}

\begin{proof}
    These details are routinely verified.
\end{proof}

\begin{corollary}
    Given any collection $\mc S$ of subsets of $X$, there is a smallest topology containing $\mc S$.
\end{corollary}

\begin{proof}
    This topology is the intersection of all the topologies that contain $\mc S$. We say that $\mc S$ generates this topology.
\end{proof}

\begin{definition}[Subbase]
    Let $\mc T$ be a topology on $X$, and let $\mc S \subseteq 2^X$, with $X = \bigcup \mc S$. We say that $\mc S$ is a subbase of $\mc T$ if $\mc T$ is the topology generated by $\mc S$.
\end{definition}

Subbases can be useful for studying topologies in the same way that a basis is useful for studying vector spaces. For example, see \autoref{prop:subbasecontinuity}.

We can explicitly characterize the topology generated by a set $\mc S$. We first state a lemma:

\begin{lemma}
    If $\mc I \subseteq 2^X$ be closed under finite intersections and $\bigcup \mc I = X$, then the topology generated by $\mc I$ consists of the arbitrary unions of sets in $\mc I$ (with union over $0$ sets being $\emptyset$).
\end{lemma}

\begin{proof}
    The only part that is not immediately obvious is the closure under finite intersections. If $\mc O_1 \bigcup_A U_\alpha$ and $\mc O_2 = \bigcup_B V_\beta$, then \[ \mc O_1 \cap \mc O_2 = \bigcup_{A, B} U_\alpha \cap V_\beta.\]
\end{proof}

\begin{proposition}
    If $\mc S \subseteq 2^X$ whose union is $X$, then the topology $\mc T$ generated by $\mc S$ consist of the arbitrary unions of $\mc I^{\mc S}$ which are defined as the set of finite intersections of sets in $\mc S$.
\end{proposition}

\begin{definition}[Base]
    A collection $\mc B \subseteq 2^X$ is called a base for a topology $\mc T$ on $X$ if $\mc T$ consists of the arbitrary unions of elements in $\mc B$.
\end{definition}

\begin{example}
    \begin{enumerate}
        \item The collection of finite intersections $\mc I^{\mc S}$ of a subbase $\mc S$.
        \item The set of open balls in a metric space.
        \item The set of intervals of the form $(-\infty, a)$ and $(b, \infty)$ in $\bb R$ under the standard topology.
    \end{enumerate}
\end{example}

Notice how the second and third examples are not closed under finite intersections. 

\begin{proposition}
    For any base $\mc B$, any finite intersection of sets in $\mc B$ must be a union of elements in $\mc B$.
\end{proposition}

\begin{proposition}
    \label{prop:subbasecontinuity}
    If $(X, \mc T_X)$ and $(Y, \mc T_Y)$ are topological spaces, and let $f \colon X \to Y$. If $\mc S_Y$ is a subbase of $\mc T_Y$, then $f$ is continuous iff for all $U \in \mc S_Y$, we have $f^{-1}(U)$ is open.
\end{proposition}

\begin{proof}
    First show that the preimages of sets in $\mc I_{\mc S_Y}$ are open, and then show that the preimages of all sets in $\mc T_Y$ are open.
\end{proof}

\section{New Topologies from Old}

\begin{definition}[Initial Topologies]
    If $X$ is a set, $(Y_\alpha, \mc T_\alpha)$ are topological spaces, and for each $\alpha$, $f_\alpha\colon X \to Y_\alpha$. Then, the initial topology is the smallest topology on $X$ that makes all $f_\alpha$ continuous.
\end{definition}

This initial topology is just the topology generated from the preimages of open sets in $Y_\alpha$ under $f_\alpha$. This gives rise to a few important methods of constructing new topological spaces from old ones.

\begin{definition}[Relative topology]
    If $X \subseteq Y$ and $(Y, \mc T_Y)$ is a topological space, and if $j$ is the inclusion map from $X$ to $Y$, then the initial topology generated by $j$ consists of the intersection of open sets with $X$, which is also called the relative (or subspace) topology. 
\end{definition}

\begin{definition}[Finite product topology]
    If $(X_1, \mc T_1)$ and $(X_2, \mc T_2)$ are topological spaces then if $X = X_1 \times X_2$. We have a couple of ``natural'' maps, $X \xrightarrow{p_1} X_1$, $X \xrightarrow{p_2} X_2$. We see that $p_1^{-1}(U) = U \times X_2$, and $p_2^{-1}(V) = X_1 \times V$, so it is easy to see that the product topology has a basis consisting of sets of the form $U \times V$, where $U$ is open in $X_1$ and $V$ is open in $X_2$.

    This trivially generalizes to any {\bfseries FINITE} number of topological spaces and their {\bfseries FINITE} cartesian product. However, when we generalize to infinite products, the following case occurs:
\end{definition}

\begin{definition}[Product topology]
    Suppose we have topological spaces $(X_\alpha, \mc T_\alpha)$ over a (possibly infinite) indexing set $A$, where $X = \prod X_\alpha$. Then, $p_\alpha$ is the natural projection $X \to X_\alpha$, and $p_\alpha^{-1}(U)$ is $\prod U_\beta$, where $U_\beta = X_\beta$ when $\beta \ne \alpha$ and $U_\alpha = U$. These sets form a subbase of this topology. Now, since topologies only allow finite intersections, then the product topology is generated by products $\prod_\alpha U_\alpha$, where only a finite number of $U_\alpha$ are allowed to be not equal to $X_\alpha$.
\end{definition}

\begin{example}
    If $X_i = \qty{0, 1}$ with the discrete topology, then $\prod_0^\infty X_i$ is compact.
\end{example}

\begin{definition}[Weak topology]
    For a normed vector space $(V, \norm{\cdot})$. If $V^*$ is the vector space of all continuous linear functionals. The initial topology generated by $V^*$ is the weak topology on $V$.    
\end{definition}

\begin{example}
    For example, if we have $V = \mc C^0([0, 1])$, with \[\phi_g\colon V \to \bb R, f \mapsto \int_0^1 f(x)g(x) \dd{x}.\] Then, the initial topology generated by $\phi_g$ defines ``a'' weak topology on $V$.
\end{example}

As opposed to a initial topology, we can define final topologies:
\begin{definition}[Final topology]
    If $(X_\alpha, \mc T_\alpha)$ are topological spaces and $f_\alpha \colon X_\alpha \to Y$, then the final topology $\mc T$ is the strongest topology making all $f_\alpha$ continuous.
\end{definition}

To construct this topology, we can consider $\qty{A \in 2^Y \mid f_\alpha^{-1}(A) \in \mc T_\alpha}$, which is a topology on $Y$ for each $\alpha$. We can just take the intersection of these topologies to obtain the strongest topology that makes all the $f_\alpha$ continuous.

\begin{example}[Quotient topology]
    Any function $f \colon X \to Y$ induces an equivalence relation with the equivalence classes being the preimages $f^{-1}(\qty{y})$. These equivalence classes partition $X$. It turns out any equivalence relation $\sim$ induces equivalence classes that partition $X$, which are denoted $X/\!\sim$. If $X$ has a topology $\mc T_X$, then the final topology from the natural mapping $f \colon X \to X/\!\sim$ is called the quotient topology on $X/\!\sim$.
\end{example}

We can construct a bunch of new, interesting topologies using quotients. First we define what it means for topological spaces to be ``the same.'' I will also start dropping the topology and use individual letters $X$ to denote the topological spaces $(X, \mc T_X)$, depending on context.

\begin{definition}[Homeomorphism]
    A function $f \colon X \to Y$ is a homeomorphism if it is an invertible continuous mapping, whose inverse is also continuous. Two spaces where there exists a homeomorphism between them are called ``homeomorphic.'' 
\end{definition}

\begin{example}[Circle]
    If $I = [0, 1]$ with the equivalence relation such that only $0 \sim 1$, then $X/\!\sim$ is homeomorphic to the $1$-sphere (circle), $S^1$. One homeomorphism from $I \to S^1$ is the mapping $t \mapsto e^{2\pi i t}.$
\end{example}

\begin{example}[Cut-off cylinder]
    If $X = [0, 2] \times [0, 1]$ is endowed with the relation $(0, r) = (2, r)$ then the quotient $X/\!\sim$ is homeomorphic to a cylinder. If we instead identified $(0, r) \sim (2, 1 - r)$ then $X/\!\sim$ will be homeomorphic to a M\"obius strip.
\end{example}

\begin{example}[Projective space]
    If $X$ is the unit sphere in $\bb R^3$, then if we define the equivalence $v \sim -v$, then $X/\!\sim$ is the real projective space $\bb R\mathrm P^2$.
\end{example}

\begin{example}
    It can be seen that homeomorphisms from an object from itself can form (quite complicated) groups. These groups are the automorphism groups $\operatorname{Aut}(X)$ of topological spaces. Recall that a group action on $X$ is a homomorphism $\alpha \colon G \to \operatorname{Aut}(X)$. Examples of group actions on spaces include $\bb Z$ acting on $\bb R$ by translation, or $\bb Z_2$ acting on $S^2$ by the antipodal map. 

    However, recall that group actions induce equivalence classes (orbits) on the topological space and thus induces a quotient topology $A/\alpha$. The quotient topology of the examples of the actions above give rise to $S^1$ and $\bb R \mathrm P^2$.
\end{example}


